\section{Modèle d'utilisation}
\label{Use Case}

\textbf{Acteur principal}: Employé de la ville de Québec ou du réseau de transport de la capitale (RTC)

\textbf{Parties prenantes}:

Le client veut une simulation dont bénis la vitesse peut être personnalisée (y compris l'arrêt ou le redémarrage), et qui respecte les conditions suivantes: 

- Les passagers et véhicules apparaissent au bon moment. 
- Il est possible d'observer le nombre de passagers dans un véhicule en tout temps
- Les véhicules doivent se déplacer visuellement durant la simulation, disparaissant arrivé à destination (sauf en cas de boucle).

\textbf{Garantie de succès}:

Temps minimal, temps maximal et temps moyen pour franchir une distance sont correctement mesurés et sauvés. En aucun cas il est possible d'avoir un circuit sans point de départ, de fin ou d'intersection.

\textbf{Utilisation classique}:

1. L'employé débute une nouvelle simulation.
2. L'employé place à la souris une série de points correspondant à des intersections et/ou arrêts d'autobus.
3. L'employé définit des circuits en sélection un point d'origine puis une série de points à franchir de manière consécutive.
4. L'employé identifie des profils de passagers avec un point d'origine, un point de destination, ainsi que les segments empruntés pour 
4. L'employé sélectionne une heure de début et de fin.
5. L'employé démarre la simulation.
6. Les temps associés à chaque segment du réseau sont sélectionné avec distribution triangulaire.
7. La position des véhicules et leur nombre de passagers se mettent à jour en suivant les trajets.
8. Pour chaque profil de passager, le temps minimal, maximal et moyen durant la simulation est calculé.

\textbf{Extensions}:

a. N'importe quand, l'employé peut interrompre la simulation, la résumer ou la recommencer.
b. N'importe quand durant la simulation, l'employé:
1. Clique sur un véhicule afin d'obtenir le nombre de passagers à l'intérieur.
2. Déplace sa souris sur la carte, affichant les coordonnées géographiques associées dans la barre d'état.
3. Zoom/Dézoom la carte.
c. N'importe quand lorsque la simulation n'est pas en court, l'employé:
1. Retire un arrêt d'un circuit avec le menu contextuel, les intersections sont alors retirées.
2. L'employé retire une intersection ou ajoute un arrêt ou une intersection avec le menu contextuel.

\textbf{Exigences spéciales}: Toutes les manipulations des éléments visuels doivent pouvoir être faites avec la souris.