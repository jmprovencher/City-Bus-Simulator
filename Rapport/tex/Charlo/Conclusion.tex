\chapter{Conclusion}
\label{Conclu}

En conclusion, notre application SimulatHEURE rejoint les demandes de notre client, en permettant la cr�ation d'un syst�me routier et l'�valuation de l'impact de la congestion routi�re sur celui-ci. Tel que demand�, l'application permet de rajouter arr�tes, stations, circuits et besoins en transport � la simulation, avec la fonctionnalit� d'enregistrer et de charger le syst�me routier actuel. L'arri�re-plan peut �galement �tre modifi� afin de pouvoir baser le syst�me routier sur une carte existante, conform�ments aux demandes du client.

SimulatHEURE va m�me au d�l� des attentes sur certains points. Par exemple, il offre une �chelle de grandeur permettant de plus facilement juger et comparer la distance entre deux sommets dans l'interface. Un autre point fort de notre programme est son syst�me de protection contre les erreurs: Il est par exemple impossible d'effacer un sommet faisant partie d'un circuit sans effacer le circuit, et il est impossible d'enregistrer ou de modifier le syst�me routier lors d'une simulation. La barre de messages de l'interface permet de communiquer directement au client si la derni�re m�thode tent�e fut un succ�s, ou au cas �ch�ant pour quelle raison il n'�tait pas possible de la compl�ter.

Une am�lioration possible pouvant �tre faite au logiciel serait de permettre l'�dition des besoins de transport, ce qui permettrait d'augmenter la productivit� lors de l'utilisation du logiciel. Dans le m�me ordre d'esprit, une autre am�lioration possible serait de permettre d'exporter les statistiques des profils de passager sous format texte ou de classeur. Pr�sentement, la seule option pr�sente est de copier les statistiques dans le presse-papier, ce qui n'est pas une solution �l�gante dans un contexte r�el. Un dernier point faible de notre programme est l'interface. Ajouter des ic�nes repr�sentant les actions courantes permettrait de rendre l'interface plus conviviale.